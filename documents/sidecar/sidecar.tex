\section{The Project}

\subsection{schema}

\begin{figure}[ht]
  \caption{diagram of different projects}
  \centering
  \includegraphics[scale=0.6]{SidecarDiagram.png}
  \label{fig:SidecarDiagram}
\end{figure}

A sidecar architecture is based on 2 containers, the main application and its addon. The objectif concists of only one process per container. In our case, the main application is the apache server and the addon is the exporter-apache. The containers share the same memory space, the same network stack, easily to deploy and manage the access from exporter-apache to apache server.
\subsection{Status On Apache}

\emph{status.conf}
\begin{apachecode}
  <Location /server-status>
  SetHandler server-status
  Order deny,allow
  Allow from all
  </Location> ExtendedStatus On>
\end{apachecode}

and  \emph{Dockerfile}

\begin{dockercode}
  FROM ubuntu:latest
  USER root
  ...
  RUN a2enmod status
  COPY status.conf /etc/apache2/mods-enabled/
  EXPOSE 8080
  USER 1001
  CMD ["/usr/sbin/apache2ctl", "-DFOREGROUND"]
\end{dockercode}

\subsection{Secret Access}

We firstly define our \emph{secret file}. If the access is based on a login/password.

\begin{yamlcode}
  apiVersion: v1
  kind: Secret
  metadata:
  name: github-secret
  namespace: sidecar
  type: kubernetes.io/basic-auth
  data:
  username: c3Bpa2U=
  password: dmFsZW50aW5l
\end{yamlcode}

\emph{username} and \emph{password} are defined with the command

\begin{bashcode}
  $ echo -n 'spike' | base64
  c3Bpa2U=
  $ echo -n 'valentine' | base64
  dmFsZW50aW5l
\end{bashcode}

and we run

\begin{bashcode}
  $ oc create -f gitlab-secret.yaml
\end{bashcode}

or if we use a \emph{ssh key},we generate the \emph{key}, create the secret, and link the secret to the right project
\begin{bashcode}
  $ ssh-keygen -C "openshift-source-builder/repo@github" \
  -f repo-at-github -N ''
  $ oc secrets new-sshauth repo-at-github \
  --ssh-privatekey=repo-at-github
$ oc secrets link builder repo-at-github
\end{bashcode}

When we create a new application based on this repository, it desn't work. We have to set the new \emph{build}

\begin{bashcode}
$ oc set build-secret --source bc/mysite repo-at-github
\end{bashcode}

 
\subsection{New Project}

Firstly, we create a new project

\begin{bashcode}
  $ oc new-project sidecar \
  --display-name='Side Car Project' \
  --description='Side Car Project'
\end{bashcode}

\subsection{New Build}

To obtain our image, we firstly
\begin{bashcode}
  $ oc new-build http://192.168.0.8:8880/spike/faye.git \
  --name faye
\end{bashcode}

But to resolve the issue based on the credential, we'll attribute the \emph{login/password} defined before and retstart the build process.

\begin{bashcode}
  $ oc set build-secret --source bc/faye github-secret
  $ oc start-build faye
\end{bashcode}

or directly
\begin{bashcode}
  $  oc new-build http://192.168.0.8:8880/spike/faye.git \
  --source-secret github-secret
  --name faye
\end{bashcode}

\subsection{New Application}

It's time to create our application

\begin{bashcode}
  $ oc new-app faye \
  --name fayeapp
  $ oc status
  $ oc expose service faye
  $ oc get pod
  $ oc get all name --selector app=cdnselect
\end{bashcode}

\section{The Side Car}

\subsection{Export}

We firstly export our \emph{project}.

\begin{bashcode}
  $ oc get --export is,bc,dc,svc -o yaml > export.yaml
\end{bashcode}

\subsection{ImageStream}
We delete \emph{resourceVersion, selfLink and uid}. In status, we keep dockerImageRepository (set to "")

\subsection{BuildConfig}
We delete \emph{resourceVersion, selfLink and uid}. We delete in spec.triggers.imageChange \emph{lastTriggeredImageID}

\subsection{DeploymentConfig}
We replace spec.template.spec.containers.image by \emph{faye} in the first container
We add in spec.template.spec.container

\begin{bashcode}
  - name: apache-exporter
  image: previousnext/apache-exporter
  command: [ "apache_exporter", \
  "-scrape_uri", \
  "http://127.0.0.1:8080/server-status/?auto" ]
  ports:
  - containerPort: 9117
\end{bashcode}

\subsection{Service}
We delete \emph{resourceVersion, selfLink and uid}.

We add in spec.ports

\begin{bashcode}
  - name: 9117-tcp
  port: 9117
  protocol: TCP
  targetPort: 9117
\end{bashcode}